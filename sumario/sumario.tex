% ****** Start of file .tex ******
%
%   This file is based on apssamp.tex, part of the APS files in the REVTeX 4.1 distribution.
%   Version 4.1r of REVTeX, August 2010
%
%   Copyright (c) 2009, 2010 The American Physical Society.
%   Samuel Balula, Pedro Ribeiro, Luís Macedo, Eduardo Neto 2013
%   See the REVTeX 4 README file for restrictions and more information.
%
% TeX'ing this file requires that you have AMS-LaTeX 2.0 installed
% as well as the rest of the prerequisites for REVTeX 4.1
%
% See the REVTeX 4 README file
% It also requires running BibTeX. The commands are as follows:
%
%  1)  latex filename.tex
%  2)  bibtex filename
%  3)  latex filename.tex
%  4)  latex filename.tex

\documentclass[%
	%preprint,
	%superscriptaddress,
	%groupedaddress,
	%unsortedaddress,
	%runinaddress,
	%frontmatterverbose, 
	%preprint,
	%showpacs,preprintnumbers,
	nofootinbib,
	%nobibnotes,
	%bibnotes,
	amsmath,amssymb,
	aps,
	%pra,
	%prb,
	%rmp,
	%prstab,
	%prstper,
	%floatfix,
	12pt,
	a4paper
]{article}

\usepackage{verbatim}                   % Apresentação de código
\usepackage{graphicx}                   % Include figure files
\usepackage{dcolumn}                    % Align table columns on decimal point
\usepackage{bm}                         % bold math
\usepackage[latin1,utf8]{inputenc}      % Tipos de caracteres
\usepackage[portuges]{babel}            % Português
\usepackage{indentfirst}                % Identação da primeira linha
\usepackage{hyperref}                   % add hypertext capabilities
\usepackage{float}                      %Fixar imagens
\usepackage{geometry}
\usepackage{fullpage}

\begin{document}

% % % % % % % % % % % % % % % % % % % % % % % % % % % % % % % % % % % % % % % % 
%%%%%%%%%%%%%%%%%%%%%%%%%%%%%%%%%% Início %%%%%%%%%%%%%%%%%%%%%%%%%%%%%%%%%%%%%%
% % % % % % % % % % % % % % % % % % % % % % % % % % % % % % % % % % % % % % % %


\title{Sumário do projecto}
\date{}
\maketitle


%%%%%%%%%%%%%%%%%%%%%%%%%%%%%%%%%% Introdução %%%%%%%%%%%%%%%%%%%%%%%%%%%%%%%%%%
\section{Identificação do grupo}
O grupo de trabalho é constituído pelos seguintes elementos:

\begin{center}
	\begin{tabular}{lll}
		Número mecanográfico	&Nome			&Curso	\\ \hline
		72735					&António Samuel Ávila Balula	&MEFT	\\
		73221					&Pedro Manuel Quintela Ribeiro	&MEFT	\\
		73633					&Luís Filipe Guedelha Macedo 	&MEFT	\\
	\end{tabular}
\end{center}

\section{Descrição do Projecto}
Os telemóveis evoluíram de simples aparelhos de comunicação para smartphones com elevado poder de computação, nos quais é possível realizar praticamente todo o tipo de tarefas. Uma das tecnologias que permitiu o sucesso dos smartphones é o ecrã táctil capacitivo, pois permite de modo intuitivo e inteligente integrar o controlo do aparelho no seu próprio ecrã, levando assim à produção de smartphones com ecrãs maiores e mais adequados a realizar tarefas de produtividade e à visualização de conteúdos multimédia num aparelho compacto.

Devido à grande procura de smartphones e tablets existente atualmente, um dos compostos necessários para fabricar estes ecrãs, o ITO (indium-tin oxide), tem sofrido uma enorme procura a nível mundial o que resulta numa elevada extração deste material, o que tem levado à subida imparável do preço deste recurso e à deplecção das reservas existentes, que se estima que terminem em 2027. Desta forma, existe interesse em desenvolver e produzir de forma económica uma alternativa à tecnologia utilizada atualmente.

Já foram testados protótipos de ecrãs táteis produzidos com vários materiais emergentes e com bons resultados, como por exemplo ecrãs fabricados com elétrodos de grafeno, nanotubos de carbono e nanofios de prata. No entanto estes protótipos não são viáveis comercialmente, principalmente devido aos elevados custos de produção destes materiais.

Propõe-se neste projeto a fabricação e teste de um condutor transparente fabricado com um híbrido entre grafeno e nanofios de prata, que se prevê que reduza a densidade necessária para a camada de nanofios de prata, o que poderá tornar um ecrã táctil fabricado com materiais alternativos comercialmente viável.

\section{Entidade Financiadora}
A entidade financiadora do projecto é a Fundação para a Ciência e Tecnologia (FCT).

\section{Planeamento da elaboração da proposta}
%Não se preocupe que no ultimo dia está tudo entregue.

\subsection{Discussão do tema e metodologia (até 10 de Abril)}

\subsection{Componente Científica (até 20 de Abril)}
\begin{itemize}
	\item Descrição técnica
	\item Plano de investigação e métodos
\end{itemize}

\subsection{Estrutura analítica do projeto (até 3 de Maio)}
\begin{itemize}
	\item Recursos Humanos
	\item Definição de indicadores previstos
	\item Calendarização
	\item Orçamentação
\end{itemize}
\subsection{Preenchimento do formulário da entidade financiadora (até 10 de Maio)}
\begin{itemize}
	\item Justificação do orçamento
\end{itemize}


%\section{Comentários Adicionais}
%Não existem


%%%%%%%%%%%%%%%%%%%%%%%%%%%%%%%%%%%%%%%%%%%%%%%%%%%%%%%%%%%%%%%%%%%%%%%%%%%%%%%%
% % % % % % % % % % % % % % % %     FIM    % % % % % % % % % % % % % % % % % % % 
%%%%%%%%%%%%%%%%%%%%%%%%%%%%%%%%%%%%%%%%%%%%%%%%%%%%%%%%%%%%%%%%%%%%%%%%%%%%%%%%

\end{document}
%end of file
