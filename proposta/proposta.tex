% ****** Start of file .tex ******
%
%   This file is based on apssamp.tex, part of the APS files in the REVTeX 4.1 distribution.
%   Version 4.1r of REVTeX, August 2010
%
%   Copyright (c) 2009, 2010 The American Physical Society.
%   Samuel Balula, Pedro Ribeiro, Luís Macedo, Eduardo Neto 2013
%   See the REVTeX 4 README file for restrictions and more information.
%
% TeX'ing this file requires that you have AMS-LaTeX 2.0 installed
% as well as the rest of the prerequisites for REVTeX 4.1
%
% See the REVTeX 4 README file
% It also requires running BibTeX. The commands are as follows:
%
%  1)  latex filename.tex
%  2)  bibtex filename
%  3)  latex filename.tex
%  4)  latex filename.tex

\documentclass[%
	%preprint,
	%superscriptaddress,
	%groupedaddress,
	%unsortedaddress,
	%runinaddress,
	%frontmatterverbose, 
	%preprint,
	%showpacs,preprintnumbers,
	nofootinbib,
	%nobibnotes,
	%bibnotes,
	amsmath,amssymb,
	aps,
	%pra,
	%prb,
	%rmp,
	%prstab,
	%prstper,
	%floatfix,
	12pt,
	a4paper
]{article}

\renewcommand\thesection{\arabic{section}.}
\renewcommand\thesubsection{\thesection\arabic{subsection}.}
\renewcommand\thesubsubsection{\thesubsection\arabic{subsubsection}.}

\newcommand*{\sectionpostamble}{}
\newcommand*{\aka}[2]{%
  \def\sectionpostamble{#2}%
  \section{#1}
}

\usepackage{titlesec}
\titleformat{\section}
  {\normalfont\Large\bfseries}{\thesection}{1em}{}
  [\normalfont\small\thesection\hspace{10pt}\sectionpostamble
  \global\let\sectionpostamble\relax]
%%%%%%%%%

\newcommand*{\subsectionpostamble}{}
\newcommand*{\subaka}[2]{%
  \def\subsectionpostamble{#2}%
  \subsection{#1}
}

\titleformat{\subsection}
  {\normalfont\large\bfseries}{\thesubsection}{2em}{}
  [\normalfont\small\thesubsection\hspace{20pt}\subsectionpostamble
  \global\let\subsectionpostamble\relax]
%%%%%%%%%

\newcommand*{\subsubsectionpostamble}{}
\newcommand*{\subsubaka}[2]{%
  \def\subsubsectionpostamble{#2}%
  \subsubsection{#1}
}

\titleformat{\subsubsection}
  {\normalfont\bfseries}{\thesubsubsection}{3em}{}
  [\normalfont\small\thesubsubsection\hspace{30pt}\subsubsectionpostamble
  \global\let\subsubsectionpostamble\relax]
%%%%%%%%%

\usepackage{verbatim}                   % Apresentação de código
\usepackage{graphicx}                   % Include figure files
\usepackage{dcolumn}                    % Align table columns on decimal point
\usepackage{bm}                         % bold math
\usepackage[latin1,utf8]{inputenc}      % Tipos de caracteres
\usepackage[portuges]{babel}            % Português
\usepackage{indentfirst}                % Identação da primeira linha
\usepackage{hyperref}                   % add hypertext capabilities
\usepackage{float}                      %Fixar imagens
\usepackage{geometry}
\usepackage{fullpage}

\begin{document}

% % % % % % % % % % % % % % % % % % % % % % % % % % % % % % % % % % % % % % % % 
%%%%%%%%%%%%%%%%%%%%%%%%%%%%%%%%%% Início %%%%%%%%%%%%%%%%%%%%%%%%%%%%%%%%%%%%%%
% % % % % % % % % % % % % % % % % % % % % % % % % % % % % % % % % % % % % % % %


\title{Sumário do projecto}
\date{}
\maketitle


%%%%%%%%%%%%%%%%%%%%%%%%%%%%%%%%%% Introdução %%%%%%%%%%%%%%%%%%%%%%%%%%%%%%%%%%
\section*{Identificação do grupo}
O grupo de trabalho é constituído pelos seguintes elementos:

\begin{center}
	\begin{tabular}{lll}
		Número mecanográfico	&Nome			&Curso	\\ \hline
		72735					&António Samuel Ávila Balula	&MEFT	\\
		73221					&Pedro Manuel Quintela Ribeiro	&MEFT	\\
		73633					&Luís Filipe Guedelha Macedo 	&MEFT	\\
	\end{tabular}
\end{center}


\aka{Identificação do projecto}{Project description}

\subaka{Domínio Científico}{Scientific Domain}
Como fica texto normal aqui no meio??
\subaka{Área Científica Principal}{Main Area}
\subaka{Área Científica Secundária}{Secondary Area}
\subaka{Acrónimo}{Acronym}
\subaka{Título do projecto (em português)}{Project title (in portuguese)}
\subaka{Título do projecto (em Inglês)}{Project title (in english)}
\subaka{Financiamento Solicitado}{Requested funding}
\subaka{Palavra-chave 1}{Keyword 1}
\subaka{Palavra-chave 2}{Keyword 2}
\subaka{Palavra-chave 3}{Keyword 3}
\subaka{Palavra-chave 4}{Keyword 4}
\subaka{Data de início do projecto}{Sarting date}
\subaka{Duração do projecto em meses}{Duration in months}

\aka{Instituições envolvidas}{Institutions and their roles}

\subaka{Instituição Proponente}{Principal Contractor}
\subaka{Instituição Participante}{Participating Institution}
\subaka{Unidade de Investigação}{Research Unit}
\subaka{Unidade de Investigação Adicional}{Adicional Research Unit}
\subaka{Instituição de Acolhimento}{Host Institution}

\pagebreak

\aka{Componente Científica}{Scientific Component}
\subaka{Sumário}{Abstract}
\subsubaka{Em português}{In Portuguese}
ola!
\subsubaka{Em inglês}{In English}
\subsubaka{Para publicação - Em português}{For publication - In Portuguese}
\subsubaka{Para publicação - Em inglês}{In English}

\subaka{Descrição Técnica}{Technical Description}
\subsubaka{Revisão da Literatura}{Literature Review}
\subsubaka{Plano e Métodos}{Plan and Methods}
\subsubaka{Tarefas}{Tasks}
\subsubaka{Calendarização e Gestão do Projecto}{Project Timeline and Management}
\subsubaka{Descrição da Estrutura de Gestão}{Description of the Management Structure}
\subsubaka{Lista de Milestones}{Milestone List}
\subsubaka{Cronograma}{Timeline}

\subaka{Referências Bibliográficas}{Bibliographic References}
\subaka{Publicações Anteriores}{Past Publications}
\subaka{Ressubmissão de projectos}{Project ressubmission}
\subsubaka{Ressubmissão?}{Ressubmission?}
?
\aka{Equipa de Investigação}{Research team}

\aka{Outros Projectos}{Other projects}

\aka{Indicadores previstos}{Expected indicators}

\aka{Orçamento}{Budget}

\aka{Justificação do orçamento}{Budget rationale}

\aka{Ficheiros Anexos}{Attachments}









%\section{Comentários Adicionais}
%Não existem


%%%%%%%%%%%%%%%%%%%%%%%%%%%%%%%%%%%%%%%%%%%%%%%%%%%%%%%%%%%%%%%%%%%%%%%%%%%%%%%%
% % % % % % % % % % % % % % % %     FIM    % % % % % % % % % % % % % % % % % % % 
%%%%%%%%%%%%%%%%%%%%%%%%%%%%%%%%%%%%%%%%%%%%%%%%%%%%%%%%%%%%%%%%%%%%%%%%%%%%%%%%

\end{document}
%end of file
